\documentclass[a4paper, notitlepage, 12pt]{scrartcl}
\author{Sebastian Baron, Simon Fiebich, Lukas Rost \\ \small{Team-ID: 00036}}
\title{Aufgabe 1 \\ \glqq Superstar\grqq  - Dokumentation}
\subtitle{37. Bundeswettbewerb Informatik 2018/19 - 1. Runde \\~\\}
\date{26. November 2018}
\usepackage[ngerman]{babel}
\usepackage[utf8]{inputenc}
\usepackage{graphicx}
\usepackage{wrapfig}
\usepackage{color}
\usepackage[dvipsnames]{xcolor}
\usepackage{hyperref}
\usepackage[top=2.5cm, bottom=1.5cm, left=2.5cm, right=2.5cm]{geometry}
\usepackage{fancyvrb}
\usepackage{caption}
\usepackage{mathtools}
\usepackage{amssymb}
\usepackage{fancyhdr}
\usepackage{lastpage}
\usepackage{pgfplots}
\pgfplotsset{compat=1.15}
\usepackage{tikz}
\usetikzlibrary{graphs,graphs.standard,positioning}

\usepackage{minted}
\fvset{breaklines=true}

\pagestyle{fancy}
\lhead{S. Baron, S. Fiebich, L. Rost, Team-ID: 00036}
\rhead{Aufgabe 1, Seite \thepage ~von \pageref{LastPage}}
\cfoot{ }

\newenvironment{longlisting}{\captionsetup{type=listing}}{}

\newmintedfile{java}{frame=single,linenos,samepage=false,firstnumber=1,rulecolor=\color{Gray},autogobble,breakafter=.u,fontsize=\small}

\begin{document}
\renewcommand{\contentsname}{\centerline{Inhaltsverzeichnis}}
 \maketitle
 \tableofcontents
 \thispagestyle{empty}
 \newpage
 \setcounter{page}{1}

\section{Lösungsidee}
Zunächst bietet es sich an, die Eingabe in einen Graphen umzuwandeln. Die Knoten entsprechen dabei den TeeniGram-Mitgliedern. Eine gerichtete Kante verläuft von einem Knoten $x$ zu einem Knoten $y$ genau dann, wenn $x$ $y$ im TeeniGram-Netzwerk folgt.
Für das in der Aufgabenstellung gegebene Beispiel ergibt sich folgender Graph: \\ \\
\begin{center}
\begin{tikzpicture}
  \graph{
  H -> S;
  H -> J;
  S -> J;
  };
\end{tikzpicture}
\end{center}
  \begin{thebibliography}{xx}
    \bibitem[1] {Src:Wiki} Wikipedia-Artikel zur Tiefensuche, \url{https://de.wikipedia.org/wiki/Tiefensuche}
    \bibitem[2] {Src:ADM} Steven S. Skiena: The Algorithm Design Manual, ISBN 978-1-84800-069-8
  \end{thebibliography}
\section{Umsetzung}
\section{Beispiele}
\section{Quellcode}
 \renewcommand{\listingscaption}{Quellcode}

 \begin{longlisting}
 \javafile{../Aufgabe1-Implementierung/src/de/lukasrost/bwinf2019/superstar/SuperstarHelper.java}
 \caption{Ein- und Ausgabe: \texttt{SuperstarHelper.java}}
 \end{longlisting}

 \begin{longlisting}
 \javafile{../Aufgabe1-Implementierung/src/de/lukasrost/bwinf2019/superstar/Graph.java}
 \caption{Implementierung des ADT Graph: \texttt{Graph.java}}
 \end{longlisting}

 \end{document}
